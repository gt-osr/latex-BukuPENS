%
% Halaman Abstrak
%y
%

\chapter*{ABSTRAK}

\par
Abstrak adalah versi ringkas dari proyek akhir yang dituliskan dengan sistematis. Abstrak seharusnya memberikan gambaran ringkas dari konten- konten yang penting dari proyek akhir. Kata-kata yang ada pada abstrak seharusnya dipilih dengan cermat dan disusun dengan jelas untuk mendeskripsikan isi proyek akhir. Abstrak memuat pembahasan berikut: Latar belakang permasalahan (sekitar 1 kalimat), Permasalahan (1-2 kalimat), Solusi yang ditawarkan (sekitar 5 kalimat), dan Kinerja (sekitar 2 kalimat). Latar belakang permasalahan diambil dari problem domain pada penelitian proyek akhir. Permasalahan menjelaskan tingkat urgensi dari problem pada penelitian proyek akhir. Solusi mendiskusikan tujuan utama pada proyek akhir, klaim orisinalitas penelitian, skala penelitian, dan mendiskusikan metodologi penyelesaian masalah. Kinerja membahas ringkasan eksperimen, hasil ujicoba dan analisis kinerja. Abstrak ditulis dalam 300-500 kata.
\\

\vspace*{0.2cm}

\noindent Kata kunci: \\ \f{Maksimum 5 frasa kata, dipisahkan oleh tanda koma}  \\


\newpage
