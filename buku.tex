%
% Template Laporan Skripsi/Thesis PENS
%
% @author  Lucky Mahendra Purba
% @version 1.0.0
% @Special thanks to Ichlasul Affan, Azhar Kurnia
%
% Dokumen ini dibuat berdasarkan
% konfigurasi LaTeX yang digunakan Lucky ketika membuat laporan skripsi
% aturan penulisan skripsi yang dikeluarkan PENS pada tahun 2025,
%

%
% Tipe dokumen adalah report dengan satu kolom.
%

\documentclass[12pt, a4paper, onecolumn, twoside, final]{report}
\raggedbottom

\usepackage{_internals/pens-book}
\UseRawInputEncoding
% Load konfigurasi khusus untuk laporan yang sedang dibuat
%-----------------------------------------------------------------------------%
% Judul Dokumen
%-----------------------------------------------------------------------------%
%
% Judul laporan.
\def\judul{Judul Karya Ilmiah Anda}
%
% Tulis kembali judul laporan namun dengan bahasa Ingris
\def\judulInggris{Your Scientific Publication Title}
%
% Jenjang Pendidikan
\def\jenjangPendidikan{[JENJANG PENDIDIKAN]}

%-----------------------------------------------------------------------------%
% Informasi Penulis
%-----------------------------------------------------------------------------%
%
% Tulis nama Anda
% Kosongkan penulisDua dan penulisTiga jika Anda melaksanakan tugas akhir/laporan secara individu
\def\penulisSatu{Nama Mahasiswa} % nama lengkap penulis pertama
\def\penulisDua{} % nama lengkap penulis kedua
\def\penulisTiga{} % nama lengkap penulis ketiga
%
% Tulis NRP Anda
% Kosongkan nrpDua dan nrpTiga jika Anda melaksanakan tugas akhir/laporan secara individu
\def\nrpSatu{NRP..........} % NRP penulis pertama
\def\nrpDua{} % NRP penulis kedua
\def\nrpTiga{} % NRP penulis ketiga
%
% Tulis Program Studi yang Anda ambil
% Kosongkan programDua dan programTiga jika Anda melaksanakan tugas akhir/laporan secara individu
\def\programSatu{[NAMA PRODI]} % program studi penulis pertama
\def\programDua{} % program studi penulis kedua
\def\programTiga{} % program studi penulis ketiga
%
% Tulis Program Studi yang Anda ambil dalam bahasa inggris
% Kosongkan programDua dan programTiga jika Anda melaksanakan tugas akhir/laporan secara individu
\def\studyProgramSatu{Your study program 1} % 1st author's study program
\def\studyProgramDua{} % 2nd author's study program
\def\studyProgramTiga{} % 3rd author's study program


%-----------------------------------------------------------------------------%
% Informasi Dosen Pembimbing & Penguji
%-----------------------------------------------------------------------------%
%
% Tuliskan pembimbing
% Untuk Kampus Merdeka: Tuliskan dosen PIC/pembimbing dari Fakultas Anda
\def\pembimbingSatu{Dosen Pembimbing1}
\def\nipPembimbingSatu{NIP............}
% S1 s.d. S3: Kosongkan jika tidak ada pembimbing kedua
% Untuk Kampus Merdeka: Tuliskan penanggung jawab/penyelia/mitra
%                       dari program Kampus Merdeka yang Anda ambil (jika ada)
\def\pembimbingDua{Dosen Pembimbing2}
\def\nipPembimbingDua{NIP............}
% S2 & S3: Kosongkan jika tidak ada pembimbing ketiga
\def\pembimbingTiga{}
\def\nipPembimbingTiga{}

%
% Tuliskan penguji
\def\pengujiSatu{Penguji Pertama Anda}
\def\nipPengujiSatu{NIP............}
\def\pengujiDua{Penguji Kedua Anda}
\def\nipPengujiDua{NIP............}
% Kosongkan jika tidak ada penguji ketiga (umumnya penguji ketiga hanya ada untuk S2)
\def\pengujiTiga{}
% Kosongkan jika tidak ada penguji keempat, kelima, atau keenam (umumnya penguji > 3 hanya ada untuk S3)
\def\pengujiEmpat{}
\def\pengujiLima{}
\def\pengujiEnam{}

\def\namaKaprodi{[NAMA KAPRODI]}
\def\nipKaprodi{NIP............}


%-----------------------------------------------------------------------------%
% Informasi Lain (Asal departemen, Tanggal, dsb.)
%-----------------------------------------------------------------------------%
%
% Tuliskan departemen dimana penulis berada
\def\departemen{[NAMA DEPARTEMEN]}
%
% Tuliskan bulan dan tahun publikasi laporan
\Var{\bulanTahun}{Bulan Tahun}
%
% Tuliskan gelar yang akan diperoleh dengan menyerahkan laporan ini
\def\gelar{[GELAR]}
%
% Tuliskan tanggal pengesahan laporan, waktu dimana laporan diserahkan ke
% penguji/sekretariat
\def\tanggalSiapSidang{Tanggal Bulan Tahun}
%
% Tuliskan tanggal keputusan sidang dikeluarkan dan penulis dinyatakan
% lulus/tidak lulus
\def\tanggalLulus{Tanggal Bulan Tahun}
%
% Tahun saat ini
\def\tahunSekarang{[TAHUN]}

%-----------------------------------------------------------------------------%
% Judul Setiap Bab
%-----------------------------------------------------------------------------%
%
% Berikut ada judul-judul setiap bab.
% Silahkan diubah sesuai dengan kebutuhan.
%
\Var{\kataPengantar}{Kata Pengantar}
\Var{\babSatu}{Pendahuluan}
\Var{\babDua}{Kerangka Berpikir}
\Var{\babTiga}{Penggunaan Lanjutan}
\Var{\babEmpat}{Struktur Template}
\Var{\babLima}{Kasus-Kasus Khusus}
\Var{\kesimpulan}{Penutup}


%-----------------------------------------------------------------------------%
% Capitalized Variables
% Anda tidak perlu mengubah apapun di bagian ini
%-----------------------------------------------------------------------------%
\Var{\Judul}{\judul}
\Var{\Type}{\type}
\Var{\PenulisSatu}{\penulisSatu}
\Var{\PenulisDua}{\penulisDua}
\Var{\PenulisTiga}{\penulisTiga}
\Var{\Fakultas}{\fakultas}
\Var{\ProgramSatu}{\programSatu}
\Var{\ProgramDua}{\programDua}
\Var{\ProgramTiga}{\programTiga}


% Daftar pemenggalan suku kata dan istilah dalam LaTeX
\include{_internals/hypeindonesia}
% Daftar istilah yang mungkin perlu ditandai
%\input{config/istilah}
% Awal bagian
\makeglossaries
\begin{document}

%
% Sampul Laporan
%
\include{_internals/sampul}
\ifodd\thechapterpagecount\forceclearchapter\fi

\addChapter{Sampul Putih}
% 
% Sampul Putih Laporan
% 
% @author  Lucky Mahendra
% @version 1.0.0
% 

% 
% Halaman sampul-putih depan PENS
%

\begin{titlepage}
  \begin{singlespace*}
    \begin{center}
      \vspace*{-3cm}
      \begin{tabular*}{\textwidth}{l@{\extracolsep{\fill}}r}
        \rule{0pt}{3.5cm}
        & \raisebox{1cm}{\textcolor{black}{\fontsize{20}{24}\bo{PROYEK AKHIR}}}
      \end{tabular*}
      
      \vspace*{3.0cm}
      % judul thesis harus dalam 14pt Times New Roman
      \large
      \bo{\Judul} \\[1.0cm]
      
      % Sesuaikan spacing agar semua informasi muat dalam satu halaman 
      \vspace*{2.5cm}
      
      % penulis dan nrp
      \large
      \ifx\blank\nrpDua
        \bo{\PenulisSatu} \\
        \bo{\nrpSatu} \\
      \else
        \bo{\PenulisSatu~/ \nrpSatu~/ \ProgramSatu}\\
        \bo{\PenulisDua~/ \nrpDua~/ \ProgramDua}\\
      \fi
      \ifx\blank\nrpTiga\else
        \bo{\PenulisTiga~/ \nrpTiga~/ \ProgramTiga}\\
      \fi

      \vspace*{1.0cm}

      \large
      \bo{DOSEM PEMBIMBING} \\
      \ifx\blank\pembimbingTiga
        \bo{\pembimbingSatu} \\
        \bo{\nipPembimbingSatu} \\
        \vspace*{0.5cm}
        \bo{\pembimbingDua} \\
        \bo{\nipPembimbingDua} \\
      \else
        \bo{\pembimbingSatu} \\
        \bo{\nipPembimbingSatu} \\
        \vspace*{0.5cm}
        \bo{\pembimbingDua} \\
        \bo{\nipPembimbingDua} \\
        \vspace*{0.5cm}
        \bo{\pembimbingTiga} \\
        \bo{\nipPembimbingTiga} \\
      \fi
      
      % Sesuaikan spacing agar semua informasi muat dalam satu halaman 
      \vspace*{4.0 cm}
      
      % informasi mengenai fakultas dan program studi
      \large
      \bo{
        PROGRAM STUDI \jenjangPendidikan \\
        \ProgramSatu \\
        \departemen\\                
      }

      \vspace*{1.0cm}

      \large
      \bo{
        POLITEKNIK ELEKTRONIKA NEGERI SURABAYA \\
        \tahunSekarang
      }
      \normalsize
    \end{center}
  \end{singlespace*}
\end{titlepage}

\ifodd\thechapterpagecount\forceclearchapter\fi

\pagenumbering{roman}

\addChapter{Lembar Pengesahan}
%
% Lembar Pengesahan Laporan
%
% @author  Lucky Mahendra
% @version 1.0.0
%

\begin{titlepage}
  \AddToHookNext{shipout/background}{
    \begin{tikzpicture}[remember picture,overlay]
      \node at (current page.center) {
        \includegraphics[width=\paperwidth, height=\paperheight]
        {assets/logo/lembar_pengesahan.png}
      };
    \end{tikzpicture}
  }
  \begin{singlespace*}
    \begin{center}

      {\fontsize{20}{24}\bo{HALAMAN PENGESAHAN}} \\[1.0cm]

      {\fontsize{14}{14}
        \bo{\Judul} \\[1.0cm]

        Oleh: \\
        \bo{\PenulisSatu} \\
        \bo{\nrpSatu} \\[1.0cm]

        % Paragraf Persyaratan Gelar
        \begin{minipage}{0.8\textwidth}
          \centering
          \bo{Proyek Akhir ini digunakan sebagai salah satu syarat untuk
            memperolah {\gelar}
            di Politeknik Elektronika Negeri Surabaya}
          \\
          \bo{\tahunSekarang}
        \end{minipage}
        \vspace{1cm}

        % Blok Tanda Tangan
        % Menggunakan \tabular untuk perataan

        \bo{Disetujui oleh:} \\
        \begin{tabular}{l l l l}
          % -- Pembimbing 1 --
          \rule{0pt}{1cm}
          Pembimbing 1 & : {\pembimbingSatu} & ( & \hspace{3cm} ) \\
                       & {\hspace{0.25cm}\nipPembimbingSatu} & & \\
          \rule{0pt}{1cm}

          % -- Pembimbing 2 --
          Pembimbing 2 & : {\pembimbingDua} & ( & \hspace{3cm} ) \\
                       & {\hspace{0.25cm}\nipPembimbingDua} & & \\
          \rule{0pt}{1cm}

          % -- Penguji 1 --
          Penguji 1    & : {\pengujiSatu} & ( & \hspace{3cm} ) \\
                       & {\hspace{0.25cm}\nipPengujiSatu} & & \\
          \rule{0pt}{1cm} %

          % -- Penguji 2 --
          Penguji 2    & : {\pengujiDua} & ( & \hspace{3cm} ) \\
                       & {\hspace{0.25cm}\nipPengujiDua} & & \\
        \end{tabular}
        \vspace{1cm}

        % Blok Mengetahui (Ketua Program Studi)
        \begin{center}
          Mengetahui, \\ %
          Ketua Program Studi Sarjana Terapan \programSatu \\ %
          Politeknik Elektronika Negeri Surabaya \\ %
          \vspace{2cm} % Spasi untuk TTD

          {\bo{\namaKaprodi}} \\ %
          NIP. ................... %
        \end{center}

        \normalsize
      \end{center}
    \end{singlespace*}
  \end{titlepage}}

\ifodd\thechapterpagecount\forceclearchapter\fi

\addChapter{Pernyataan Orisinalitas}
% 
% Pernyataan Orisinalitas
% 
% @author  Lucky Mahendra
% @version 1.0.0
% 

% 
% 
% PERNYATAAN ORISINALITAS (sesuai Panduan PENS)
% 

\begin{titlepage}
	
    \begin{center}
      {\fontsize{20}{24}\bo{PERNYATAAN ORISINALITAS}} \\
      \vspace{1cm}
    \end{center}

    % Rata kiri untuk paragraf
    \par
    Dengan ini saya menyatakan bahwa bagian atau keseluruhan proyek akhir ini:

    \vspace{0.2cm}
    \begin{enumerate}
    \item Adalah hasil karya sendiri dan tidak mengandung unsur plagiat dari pihak lain,
    \item Tidak pernah diajukan untuk mendapatkan gelar akademis pada suatu Perguruan Tinggi,
    \item Tidak pernah dipublikasikan atau ditulis oleh pihak lain
    \item Mencantumkan rujukan dan kutipan dengan jujur dan benar terhadap sumber referensi lain yang menunjang pembahasan pada proyek akhir ini.
    \end{enumerate}
    \vspace{0.2cm}

    \par \hspace*{1cm}
    Apabila ditemukan bukti bahwa pernyataan saya di atas tidak benar, maka saya bersedia menerima sanksi sesuai dengan ketentuan yang berlaku di Politeknik Elektronika Negeri Surabaya.

    \vspace{2cm}

    % Blok Tanda Tangan

    \par
    % \hspace* untuk memaksa spasi dari kiri
    % \mbox untuk membuat kotak tak terlihat agar teks rata kanan
    \hspace*{\fill} \mbox{Surabaya, \tanggalSiapSidang}

    \vspace{2cm} % Spasi untuk TTD

    % Rata kanan untuk Nama dan NRP
    % \bo{} membuat teks tebal
    \hspace*{\fill} \mbox{\bo{\penulisSatu}} \\ % Nama dari settings.tex
    \hspace*{\fill} \mbox{\bo{\nrpSatu}} % NRP dari settings.tex

\end{titlepage}

\ifodd\thechapterpagecount\forceclearchapter\fi

\addChapter{Penyataan HakCipta}
% 
% Pernyataan Hak Cipta
% 
% @author  Lucky Mahendra
% @version 1.0.0
% 

% 
% 
% PERNYATAAN HAK CIPTA (sesuai Panduan PENS)
% 

\begin{titlepage}
	\thispagestyle{plain}
	
    \begin{center}
      {\fontsize{20}{24}\bo{PERNYATAAN PENGALIHAN HAK}} \\[0.5cm]
      {\fontsize{20}{24}\bo{CIPTA}} \\
      \vspace{1cm}
    \end{center}

    % Rata kiri untuk paragraf
    \par \noindent
    Dengan ini, saya yang bertanda tangan di bawah ini:

    \begin{tabular}{@{} l @{\hspace{0.5cm}:\hspace{0.5cm}} l}
      \rule{0pt}{1em}
      Nama & \bo{\penulisSatu} \\
      \rule{0pt}{1em}
      NRP & \bo{\nrpSatu} \\
      \rule{0pt}{1em}
      Judul Proyek Akhir & “\bo{\judul}” \\
      \rule{0pt}{1em}
      Tanggal & Surabaya, \tanggalSiapSidang
    \end{tabular}

    \par \noindent
    menyatakan bahwa saya selaku penulis (dan/atau mewakili seluruh penulis) secara sadar dan sukarela mengalihkan hak cipta ( copyright ) atas proyek akhir tersebut kepada Politeknik Elektronika Negeri Surabaya. \\
    Demikian pernyataan ini saya buat dengan sebenar-benarnya dan tanpa paksaan dari pihak mana pun.
    \vspace{2cm}



    % Blok Tanda Tangan
    \par \noindent
    \mbox{Hormat saya,}
    \vspace{1cm}

    \par \noindent
    \mbox{[Materai]}
    \vspace{1cm}

    \par \noindent
    \mbox{\bo{\penulisSatu}}

\end{titlepage}

\ifodd\thechapterpagecount\forceclearchapter\fi

\pagestyle{first-pages}
%
% Abstrak
%
\addChapter{Abstrak}
%
% Halaman Abstrak
%y
%

\chapter*{ABSTRAK}

\par
Abstrak adalah versi ringkas dari proyek akhir yang dituliskan dengan sistematis. Abstrak seharusnya memberikan gambaran ringkas dari konten- konten yang penting dari proyek akhir. Kata-kata yang ada pada abstrak seharusnya dipilih dengan cermat dan disusun dengan jelas untuk mendeskripsikan isi proyek akhir. Abstrak memuat pembahasan berikut: Latar belakang permasalahan (sekitar 1 kalimat), Permasalahan (1-2 kalimat), Solusi yang ditawarkan (sekitar 5 kalimat), dan Kinerja (sekitar 2 kalimat). Latar belakang permasalahan diambil dari problem domain pada penelitian proyek akhir. Permasalahan menjelaskan tingkat urgensi dari problem pada penelitian proyek akhir. Solusi mendiskusikan tujuan utama pada proyek akhir, klaim orisinalitas penelitian, skala penelitian, dan mendiskusikan metodologi penyelesaian masalah. Kinerja membahas ringkasan eksperimen, hasil ujicoba dan analisis kinerja. Abstrak ditulis dalam 300-500 kata.
\\

\vspace*{0.2cm}

\noindent Kata kunci: \\ \f{Maksimum 5 frasa kata, dipisahkan oleh tanda koma}  \\


\newpage

%
% Halaman Abstract
%y
%

\chapter*{ABSTRACT}

\par
\f{Abstrak ditulis dalam bahasa Inggris, sebagai terjemahan dari bagian abstrak sebelumnya, dan ditulis menggunakan  format huruf italic (miring).
}
\\

\vspace*{0.2cm}

\noindent Key Words: \\ \f{Maksimum 5 frasa kata, dipisahkan oleh tanda koma}  \\


\newpage

%
% Halaman Kata Pengantar
%
%

\chapter*{Kata Pengantar}

\par
Penulis dapat dapat memberikan kata pengantar sebagai prakata penulis dalam mengawali dalam penulisan buku proyek akhir. Penulis dapat menyampaikan cerita-cerita dan kesan saat melakukan penelitian sampai penulisan buku proyek akhir, memberikan apresiasi dan ucapan terima kasih kepada pihak-pihak yang telah membantu penulis dalam penelitian dan penulisan buku proyek akhir, ataupun memberikan motivasi dan pesan kepada pembaca. Kata pengantar seharusnya menggunakan kata-kata yang baik dan gaya bahasa formal. Pada  bagian kata pengantar dapat ditambahkan ucapan terima kasih kepada pihak-pihak yang berkontribusi dalam pelaksanaan proyek akhir.



\newpage


\end{document}
